% !TEX program = xelatex
\documentclass[b5paper, 11pt, openleft]{memoir}

\usepackage{indentfirst}

%%% Font setup
\usepackage[no-math]{fontspec}
\setmainfont{Zilla Slab}
\setmonofont[Scale = MatchLowercase]{Iosevka Custom Medium Condensed}
\XeTeXlinebreaklocale="en_EN"
\XeTeXlinebreakskip=0pt plus 3pt
\emergencystretch=1em

\usepackage{mathtools}
\usepackage[math-style = ISO, mathrm = sym, warnings-off = {mathtools-colon}]{unicode-math}
\setmathfont[Scale = MatchLowercase]{Concrete-Math.otf}
\noDisplayskipStretch
\setoperatorfont\symscr
\everymath{\displaystyle}

%%% Stylings
% Page Layout and Margins
\setsecnumdepth{subsection}
\settocdepth{subsection}
\setlrmarginsandblock{4cm}{1.5cm}{*}
\setulmarginsandblock{3cm}{2.5cm}{*}
\setlength{\headheight}{30pt}
\setlength{\parindent}{1.5cm}
\setlength{\parskip}{0.3em}
\setlength{\beforechapskip}{20pt}
\renewcommand{\arraystretch}{1}
\allowdisplaybreaks
\setSpacing{1.5}
\checkandfixthelayout

% Footnotes
\usepackage{fancyhdr}
\pagestyle{fancy}
\fancyhead[LE]{\textbf{\thepage ~ $\big\vert$ \leftmark}}
\fancyhead[RO]{\textbf{\rightmark ~ $\big\vert$ \thepage}}
\fancyhead[RE, LO, C]{}

% Styling Figures
\usepackage{multicol, caption}
\setlength{\columnseprule}{1pt}
\def\columnseprulecolor{\color{lightgray}}
\DeclareCaptionLabelSeparator{pipe}{ $\vert$ }
\captionsetup{
    labelfont = {bf},
    font = {small, sc},
    width = 0.6\textwidth,
    labelsep = pipe,
    figurename = \textbf{Fig. }
}

% Styling Titles
\renewcommand{\partnamefont}{\LARGE\bfseries\scshape\centering}
\renewcommand{\partnumfont}{\LARGE\bfseries\scshape\centering\MakeUppercase}
\renewcommand{\midpartskip}{\par\rule{1in}{0.5pt}\vspace{1em}\par}
\renewcommand{\printparttitle}{\HUGE\bfseries\scshape\centering}
\renewcommand{\afterpartskip}{\relax}
\chapterstyle{veelo}
    \renewcommand*{\printchapternum}{%
    \makebox[0pt][l]{%
    \hspace{.8em}%
    \resizebox{!}{\beforechapskip}%
    {\chapnumfont \thechapter}%
    \hspace{.8em}%
    \rule{2\midchapskip}{\beforechapskip}%
    }%
}

% Packages
\usepackage[dvipsnames]{xcolor}
\usepackage{subcaption, graphicx, pdfpages, float, wrapfig}
\usepackage{minted}
\usemintedstyle{base16-unikitty-light}
\setminted{
    frame = lines,
    bgcolor = lightgray!20,
    linenos,
	breaklines,
	fontsize = \small
}
\usepackage{csquotes}
\graphicspath{{figures/}}
\usepackage[inline]{enumitem}

\usepackage[colorlinks, allcolors = blue]{hyperref}
\usepackage{cleveref}

%%% Mathematical packages 
\usepackage[]{siunitx}
\usepackage{physics2}
\usepackage{derivative}
\usephysicsmodule{ab, ab.braket, nabla.legacy, op.legacy}
\usephysicsmodule{ab.legacy}
\usepackage[makeroom]{cancel}
% Proofs
\usepackage{amsthm}
\usepackage{tcolorbox}
\tcbuselibrary{breakable, theorems, skins}
\newtcbtheorem[auto counter, crefname = {theorem}{theorems}, Crefname = {Theorem}{Theorems}]{theorem}{Theorem}{
    coltitle = black,
    sharp corners, frame hidden, enhanced, colback = lightgray!10, breakable,
    borderline west = {3pt}{-3pt}{lightgray},
    detach title = true,
    fonttitle = \bfseries, before upper = {\tcbtitle\quad}
}{theorem}
\newtcbtheorem[auto counter, crefname = {axiom}{axioms}, Crefname = {Axiom}{Axioms}]{axiom}{Axiom}{
    sharp corners, colback = lightgray!40, colframe = darkgray, breakable
}{axiom}
\newtcbtheorem[auto counter, crefname = {definition}{definition}, Crefname = {Definition}{Definition}]{df}{Definition}{
    sharp corners, colback = lightgray!40, colframe = darkgray, breakable
}{df}
\newtcbtheorem[auto counter, number within = section]{exmp}{Example}{
    colback = lightgray!40, colframe = darkgray, breakable
}{exmp}
\newtcbtheorem[auto counter, number within = chapter, crefname = {remarks of chapter }{remarks of chapter }, Crefname = {Remarks}{Remarks}]{remark}{Remarks on chapter }{
    colback = lightgray!10, colframe = black, breakable
}{remark}

%%% Mathematical commands
% Geometry
\let\line\overline
% Mathematical constants
\newcommand{\e}{\symrm{e}}
\newcommand{\im}{\symrm{i}}
\newcommand{\cpi}{\symrm{\pi}}
\DeclareMathOperator*{\ssum}{\symrm{\Sigma}}
\DeclareMathOperator*{\Proj}{\symrm{Proj}}
\DeclareMathOperator*{\sgn}{\symrm{sgn}}
% Vector notations
\newcommand{\vv}[1]{\pmb{\symrm{#1}}}
\newcommand{\vdot}{\pmb{\cdot}}
\newcommand{\conj}{^{\ast}}
\newcommand{\dagr}{^{\dag}}
\newcommand{\trnsp}{^{\intercal}}
\newcommand{\iden}{\symbb{I}}
\newcommand{\uv}[1]{\hat{\vv{e}}_{#1}}
\newcommand{\tensor}{\otimes}
\newcommand{\bmat}[1]{
	\begin{bmatrix}
		#1
	\end{bmatrix}
}
\newcommand{\ihat}{\hat{\i}}
\newcommand{\jhat}{\hat{\j}}
\newcommand{\khat}{\hat{k}}
\newcommand{\xhat}{\hat{\vv{x}}}
\newcommand{\yhat}{\hat{\vv{y}}}
\newcommand{\zhat}{\hat{\vv{z}}}
\newcommand{\rhat}{\hat{\vv{r}}}
\newcommand{\nhat}{\hat{\vv{n}}}
\newcommand{\that}{\hat{\vv{\theta}}}
\newcommand{\phat}{\hat{\vv{\rho}}}
\newcommand{\eflux}{\symrm{\Phi}_E}
\newcommand{\mflux}{\symrm{\Phi}_B}
\newcommand{\sint}{\int_{\mathcal{S}}}
\newcommand{\aint}{\int_{\mathcal{A}}}
\newcommand{\vint}{\int_{\mathcal{V}}}
\newcommand{\cint}{\int_{\mathcal{C}}}
\newcommand{\bperm}{\symrm{\mu}_0}
\newcommand{\eperm}{\symrm{\varepsilon}_0}
\newcommand{\rc}{{{\mbox{$\resizebox{1.2ex}{1.15ex}{\includegraphics[trim= 1em 0 14em 0,clip]{fonts/scriptr.pdf}}$}}}}
\newcommand{\brc}{{{\mbox{$\resizebox{1.2ex}{1.15ex}{\includegraphics[trim= 1em 0 14em 0,clip]{fonts/boldcursiver.pdf}}$}}}}
\newcommand{\rchat}{{{\mbox{$\hat\rcurs$}}}}
\newcommand{\peval}[1]{\left(\left.#1\right.\right\rvert}
% Differences
\DeclareMathOperator{\kdel}{\symrm{\delta}}
\DeclareMathOperator{\ddel}{\symrm{\delta}}
\newcommand{\Dd}{\symrm{\Delta}}
% Physics quantities symbols
\newcommand{\lagr}{\mathcal{L}}
\newcommand{\haml}{\mathcal{H}}
\newcommand{\hilb}{\mathcal{E}}
\newcommand{\emf}{\mathcal{E}}
% Calculus notations
\newcommand{\appr}{\rightarrow}
\newcommand{\alc}[2][0.3]{&\parbox[c]{#1\textwidth}{#2}}
\newcommand{\pintm}[1]{\mathcal{D}[#1]}
% Mathematical conjunctions and expressions
\newcommand{\mathand}{\quad\textrm{and,}\quad}
\newcommand{\mathor}{\quad\textrm{or,}\quad}
\newcommand{\mathif}{\quad\textrm{if}\quad}
\newcommand{\mathiff}{\quad\textrm{\emph{iff}}\quad}
\newcommand{\maththerefore}{\therefore\emquad}
\newcommand{\ifft}{\emph{iff}}
% Notational commands
\newcommand{\flatfrac}[2]{#1\fracslash#2}
% Column types
\newcolumntype{C}{>{$}c<{$}}
\newcolumntype{L}{>{$}l<{$}}
\newcolumntype{R}{>{$}r<{$}}

%%% Type commands
\newcommand{\conclusion}{\section{Conclusion for Chapter \thechapter}}
\newcommand{\formula}{\section{Formula from Chapter \thechapter}}
\newcommand{\prelude}[1]{
    \chapter*{Prelude: #1}
    \addcontentsline{toc}{chapter}{Prelude: #1}
}
\newcommand{\prerequisites}[1]{\textbf{Prerequisites:}~\emph{#1}}

% Bibliographies
\usepackage[
    backend = biber,
    style = phys,
    sorting = anyvt
]{biblatex}
\addbibresource{bibfile.bib}

\usepackage[inkscapeversion = 1, inkscapelatex = true]{svg}
\svgpath{{code/}}
% Indices
\usepackage{imakeidx}
\makeindex

\begin{document}

\frontmatter
\title{
	\vspace{-5em}
	\textbf{
		AI Builders: 2D Fluid simulation framework via the Lattice-Boltzmann method with conditional optimizers
	}
}
\author{Puripat Thumbanthu, Kunakorn Chaiyara}
\date{Started: November 11, 2024; Revision: \today}
\maketitle

\tableofcontents*

\mainmatter

\chapter{Introduction}

\section{Backgrounds}

This project, submitted to the \texttt{AI Builders X ESCK} program, originated from a series of questions.
\begin{itemize}
	\item What's the best way to blow on a liquid filled spoon to cool it?
	\item Given a room, what's the best place to place an air conditioner, and what direction must it face?
	\item What's the best place for a cooling fan in a CPU?,
\end{itemize}
etc. These problems are a set of problems that all fall in optimization problems in fluids:
\begin{quote}
	\emph{"Given an imposed boundary condition on a system containing fluids, a boundary condition that's free to move, and a certain function, find the boundary condition that optimizes the function."}
\end{quote}
E.g., in the first problem, the imposed boundary condition is the shape of the room; the free boundary condition is the placement and angle of the air conditioner, and the function is the time until the room reaches thermal equilibrium.

Due to the ten weeks time limit imposed by the \texttt{AI Builders X ESCK} program. We've decided to simplify various parts of the problem to make it more fathomable.

\section{Problem statement and overview of solution}

The problem that we've selected to tackle is the second problem due to phase homogeneity: \enquote{given a room, what's the best place to place an air conditioner, and what direction must it face.} Due to complexity in three-dimensions, we've decided to simplify the problem to a room with boundaries in two-dimensions, and only allow the air conditioner to exist on a line around the border of the room.

The variables that are used in this problem is as follows:
\begin{enumerate}[noitemsep]
	\item \textbf{The function needed for optimization}---the time until equilibrium
	\item \textbf{Free boundary condition}---placement of the air conditioner, represented as a velocity boundary condition
	\item \textbf{Imposed boundary condition}---shape of the room
\end{enumerate}
It's then solved as follows:
\begin{enumerate}[noitemsep]
	\item Build a fluid simulator with wall and velocity boundary condition,
	\item Input the shape of the room, and the strength of the air conditioner,
	\item Find the optimal air conditioner using gradient descent.
\end{enumerate}

Originally, we planned to use the \texttt{OpenFOAM} simulator, as it's commonly used by researchers in computational fluid dynamics. However, the learning curve is too steep for just ten weeks. There's no clean way to connect the data from \texttt{OpenFOAM} into \texttt{Python} for post-processing. Most importantly, there aren't many great resources out there. So, we've decided to build our own simulation and optimization algorithm from scratch using one of the most accessible methods to do fluid simulation: the Lattice-Boltzmann method.

\section{Overview of the Lattice-Boltzmann method}

The Lattice-Boltzmann method is a fluid simulation method that doesn't require discretization of the Navier-Stokes equation. Instead, it models fluids as a collection of particles in a lattice filled with cells. In each step of the simulation, the particle moves from its own cell to its adjacent cells. Then, it interacts inside the cell through self-collisions. This cell-interpretation allow the derivation of the macroscopic fluid properties, e.g., density and velocity, to be derived from the particle distributions in each lattice directly. The process includes
\begin{enumerate}
	\item \textbf{Streaming}---particles move into adjacent cells
	\item \textbf{Collisions}---the densities in each cell is adjusted towards equilibrium inside the cell.
\end{enumerate}

This method is very viable for parallel computing, making it very ideal for implementation in \texttt{NumPy}. However, it is numerically unstable for high-speed fluid flows near or above the speed of sound. Since we're not dealing with particles moving that fast, we should be fine.

Even though the Lattice-Boltzmann method is stable for the most part, it still has some numerical instabilities around boundary conditions especially anything to do with circles. These will become a problem in gradient descent, in which we have to implement an algorithm to work around these instabilities.

\subsection{Representation}

\paragraph{Coordinate convention} Since \texttt{NumPy} indexes the $y$-axis (vertically) before the $x$-axis (horizontally), all pairs of coordinates from now on is to be read as $(y, x)$, not $(x, y)$

A fluid simulation with resolution $N \times M$ illustrated in \cref{fig:fluid-lattice}, is represented as a rectangular lattice with $N \times M$ cells. Each cell in the lattice contains nine cell-invariant unit vectors, $\vv{e}_0$ to $\vv{e}_8$, which represents the eight possible direction that the fluid can travel in. The value for these vectors, respective to the Cartesian representation is given in \cref{tab:unit-vectors}

\begin{table}[ht]
	\centering
	\begin{tabular}{C | C}
		\textrm{Unit vector} & \textrm{Representation} \\
		\hline
		\vv{e}_0             & \vv{0}                  \\
		\vv{e}_1             & \xhat                   \\
		\vv{e}_2             & \yhat                   \\
		\vv{e}_3             & -\xhat                  \\
		\vv{e}_4             & -\yhat                  \\
		\vv{e}_5             & \xhat + \yhat           \\
		\vv{e}_6             & -\xhat + \yhat          \\
		\vv{e}_7             & -\xhat - \yhat          \\
		\vv{e}_8             & \xhat - \yhat
	\end{tabular}
	\caption{Unit vectors used in a cell of fluid}
	\label{tab:unit-vectors}
\end{table}

\begin{figure}[ht]
	\centering
	\includegraphics{fluid-lattice.pdf}
	\caption{Lattice of fluid}
	\label{fig:fluid-lattice}
\end{figure}

From the set of vectors $\vv{e}_0, \dots, \vv{e}_8$ inside each cell, one respectively assign another set of vectors $\vv{f}_0, \dots, \vv{f}_8$. These vectors are scaled version of the unit vectors, i.e.,
\begin{equation}
	\vv{f}_i = f_i\vv{e}_i,
\end{equation}
where the scalar $f_n$ represents the amount of fluid that's moving in the direction $\vv{e}_n$. From this representation alone, the density, momentum, and speed of the fluid at a certain point $(n, m)$ can be found. The density of fluid at the cell $(n, m)$, $\rho(n, m)$, is the sum from of all $f_i$'s inside the cell:
\begin{equation}
	\rho(n, m) \equiv \sum_if_i(n, m).
\end{equation}
The momentum density, $\vv{U}(n, m)$, traditionally given by the product between velocity and mass, can be calculated as the sum of product between $f_n$ and their respective unit vectors:
\begin{equation}
	\vv{U}(n, m) \equiv \sum_nf_n\vv{e}_n.
\end{equation}
The velocity density at a certain cell is just the ratio between the momentum density and the fluid density:
\begin{equation}
	\vv{u}(n, m) \equiv \frac{\vv{U}(n, m)}{\rho(n, m)} = \frac{\sum_nf_n\vv{e}_n}{\rho}.
\end{equation}

\subsection{Self-collision step}

The self-collision step represents the relaxation of fluid that happens inside a cell. In each of the fluid vectors $\vv{f}_0,\dots,\vv{f}_8$, a corresponding equilibrium vector is assigned by
\begin{equation}
	\vv{E}_i(n, m) = w_i\rho\ab(1 + 3\vv{e}_i \cdot \vv{u}(n, m) + \frac{9}{2}\ab(\vv{e}_i \cdot \vv{u}(n, m))^2 - \frac{3}{2}|\vv{u}(n, m)|^2)\vv{e}_i
\end{equation}
where $w_i$ is a weighting factor:
\begin{gather}
	w_i = \begin{cases}
		\frac{4}{9}  & \textrm{if} ~ i = 0,          \\
		\frac{1}{9}  & \textrm{if} ~ i = 1, 2, 3, 4, \\
		\frac{1}{36} & \textrm{if} ~ i = 5, 6, 7, 8.
	\end{cases}
\end{gather}
The corresponding equilibrium scalar $E_i(n, m)$, is given by the relation
\begin{equation}
	\vv{E}_i(n, m) = E_i(n, m)\vv{e}_i.
\end{equation}

However, a fluid cannot possibly reach its own equilibrium in just one step; therefore, the Lattice-Boltzmann adjusts the fluid vector to approach the equilibrium vector. This behavior is captured by the relaxation time $\tau$. For the set of fluid vector positioned at the cell $(n, m)$ at time $t$, $f_i(n, m; t)$, the fluid vector at the next time step, $t + \Dd{t}$ is given by
\begin{equation}
	f_i(n, m; t + \Dd{t}) = f_i(n, m; t) + \frac{1}{\tau}\ab(E_i(n, m; t) - f_i(n, m; t)),
\end{equation}
and that
\begin{equation}
	\vv{f}_i(n, m; t + \Dd{t}) = f_i(n, m; t + \Dd{t})\vv{e}_i.
\end{equation}

\subsection{Streaming step}

Using the vector that's adjusted to the equilibrium from the self-collision step, that vector is streamed to the adjacent cells, given by
\begin{equation}
	\vv{f}_i\ab(n + (\yhat \vdot \vv{e}_i), m + (\xhat \vdot \vv{e}_i)) = \vv{f}_i(n, m; t + \Dd{t}).
\end{equation}
Basically, this equation moves the fluid from one cell to the other as illustrated in \cref{fig:streaming-step}
\begin{figure}
	\centering
	\begin{subfigure}{0.45\textwidth}
		\centering
		\includegraphics{streaming-before.pdf}
		\caption{Before}
	\end{subfigure}
	\begin{subfigure}{0.45\textwidth}
		\centering
		\includegraphics{streaming-after.pdf}
		\caption{After}
	\end{subfigure}
	\caption{Fluid vectors before and after the streaming step highlighted in color. Gray vectors are not considered.}
	\label{fig:streaming-step}
\end{figure}

\subsection{Boundary conditions}

There are two boundaries condition that needs to be implemented in this problem: wall and velocity. Since we want this to be a complete framework for two-dimensional fluid simulation, we also implemented the density boundary condition for completeness' sake.

\paragraph{Directional density boundary condition} This boundary condition can be achieved by explicitly setting the value of $f_0$ to $f_8$ after a complete simulation step.

\paragraph{Wall boundary condition} This boundary condition is sometimes referred off as the bounce-back boundary condition. If the fluid from an adjacent cell is streamed into a wall located at $(n, m)$, the wall simply reflects the fluid vector back:
\begin{equation}
	f_j\ab(n - (\e_i \vdot \yhat), m - (\e_i \vdot \xhat)) = f_i(n, m)
\end{equation}
where
\begin{equation}
	j = \begin{cases}
		i + 2 & \textrm{if} ~ i = 1, 2, 5, 6, \\
		i - 2 & \textrm{if} ~ i = 3, 4, 7, 8.
	\end{cases}
\end{equation}
Since the fluid cannot possibly stream into the center of the wall, $j$ doesn't have to be defined at $i = 0$. \cite{adams-no-date}

\paragraph{Wall-velocity boundary condition} Given a wall that's located at position $(n, m)$, and an exposed fluid cell located at position $\ab(n + (\vv{e}_a\vdot\yhat), m + (\vv{e}_b\vdot\xhat))$, the velocity boundary condition can be defined by two variables: velocity along $\vv{e}_a$, and along its clockwise perpendicular, $\vv{e}_b$ where
\begin{equation}
	b = \begin{cases}
		a + 3 & \textrm{if} ~ a = 1,       \\
		a - 1 & \textrm{if} ~ a + 2, 3, 4.
	\end{cases}
\end{equation}
Here, we define the other directions that are relative to direction $a$:
\begin{gather}
	\alpha = \begin{cases}
		a + 2 & \textrm{if} ~ a = 1, 2, \\
		a - 2 & \textrm{if} ~ a = 3, 4,
	\end{cases} \\
	\beta = \begin{cases}
		a + 1 & \textrm{if} ~ a = 1, 2, 3, \\
		a - 3 & \textrm{if} ~ a = 4,
	\end{cases} \\
	A = \begin{cases}
		a + 7 & \textrm{if} ~ a = 1,       \\
		a + 3 & \textrm{if} ~ a = 2, 3, 4,
	\end{cases} \\
	B = \begin{cases}
		a + 6 & \textrm{if} ~ a = 1, 2, \\
		a + 2 & \textrm{if} ~ a = 3, 4,
	\end{cases} \\
	C = \begin{cases}
		a + 5 & \textrm{if} ~ a = 1, 2, 3, \\
		a - 1 & \textrm{if} ~ a = 4,
	\end{cases} \\
	D = a + 4.
\end{gather}
These directions live on a grid relative to direction $a$ as illustrated in \cref{fig:relative-direction}.
\begin{figure}[ht]
	\centering
	\includegraphics{relative-direction.pdf}
	\caption{Directions relative to the convention given by the wall-velocity boundary condition. The shaded region is the wall, and the cell with vector arrows is the target cell in which wall-velocity boundary condition is applied.}
	\label{fig:relative-direction}
\end{figure}

Given that
\begin{equation}
	\vv{u}_a\ab(n + (\vv{e}_a\vdot\yhat), m + (\vv{e}_b\vdot\xhat)) \mathand \vv{u}_b\ab(n + (\vv{e}_a\vdot\yhat), m + (\vv{e}_b\vdot\xhat))
\end{equation}
is fixed by the boundary condition, the surrounding velocities in the cell $\ab(n + (\vv{e}_a\vdot\yhat), m + (\vv{e}_b\vdot\xhat))$ can be updated as follows: \cite{zou-1997}
\begin{gather}
	\vv{f}_a = \vv{f}_{\alpha} + \frac{2}{3}\rho\ab(n + (\vv{e}_a\vdot\yhat), m + (\vv{e}_b\vdot\xhat))\vv{f}_a, \\
	\vv{f}_A = \vv{f}_{C} - \frac{1}{2}\ab(\vv{f}_b - \vv{f}_{\beta}) + \rho\ab(n + (\vv{e}_a\vdot\yhat), m + (\vv{e}_b\vdot\xhat))\ab(\frac{\vv{u}_b}{6} + \frac{\vv{u}_a}{6}), \\
	\vv{f}_D = \vv{f}_{B} - \frac{1}{2}\ab(\vv{f}_b - \vv{f}_{\beta}) + \rho\ab(n + (\vv{e}_a\vdot\yhat), m + (\vv{e}_b\vdot\xhat))\ab(- \frac{\vv{u}_b}{6} + \frac{\vv{u}_a}{6}).
\end{gather}
For the rest of the directions, use the wall boundary condition (bounce-back) to calculate the fluids vector.

\chapter{Model building}

\section{Components of simulation}

\section{Equilibrium step}

\section{Colliding step}

\section{Boundary conditions}

\section{Equilibrium detection}

\chapter{Framework structure}
\label{sec:model-structure}

As said, we shall enforce the object-oriented paradigm of \texttt{Python} by turning everything into an object. Firstly, a class that initializes the simulation called \texttt{Simulation} that's initialized with the following variables
\begin{itemize}[noitemsep]
	\item \texttt{yResolution: int}
	\item \texttt{xResolution: int}
	\item \texttt{initCondition: np.array}
	\item \texttt{wallboundary: WallBoundary}
	\item \texttt{densityBoundaries: list = []}
	\item \texttt{velocityBoundaries: list = []}
	\item \texttt{relaxationTime: float = 0.8090}
	\item \texttt{initialStep: int = 0}
\end{itemize}
Some class variables is also baked into the class
\begin{minted}{python}
class Simulation:
    unitVect = np.array(
        [[0, 0], [1, 0], [0, 1], [-1, 0], [0, -1], [1, 1], [-1, 1], [-1, -1], [1, -1]]
    )
    unitX = np.array([0, 1, 0, -1, 0, 1, -1, -1, 1])
    unitY = np.array([0, 0, 1, 0, -1, 1, 1, -1, -1])
    weight = np.array(
        [4 / 9, 1 / 9, 1 / 9, 1 / 9, 1 / 9, 1 / 36, 1 / 36, 1 / 36, 1 / 36]
    )
    latticeSize = 9
    reflectIndices = {0: 0, 1: 3, 2: 4, 3: 1, 4: 2, 5: 7, 6: 8, 7: 5, 8: 6}
\end{minted}
Then, the following class variables are calculated:
\begin{minted}{python}
self.yResolution = yResolution
self.xResolution = xResolution
self.yIndex = np.arange(yResolution)
self.xIndex = np.arange(xResolution)
self.initCondition = copy.deepcopy(initCondition)
self.fluid = copy.deepcopy(initCondition)
self.lastStepFluid = initCondition
self.relaxationTime = relaxationTime
self.wallBoundary = wallBoundary
self.wallBoundary.generateIndex()
self.fluid[self.wallBoundary.boundary, :] = 0
self.pressureBoundaries = pressureBoundaries
self.velocityBoundaries = velocityBoundaries
self.step = initialStep

self.density = np.sum(self.fluid, axis=2)
self.momentumY = np.sum(self.fluid * Simulation.unitY, axis=2)
self.momentumX = np.sum(self.fluid * Simulation.unitX, axis=2)
self.speedY = self.momentumY / self.density
self.speedX = self.momentumX / self.density
self.speedY = np.nan_to_num(self.speedY, posinf=0, neginf=0, nan=0)
self.speedX = np.nan_to_num(self.speedX, posinf=0, neginf=0, nan=0)
\end{minted}
In this class contains the following main functions that are used to simulate the fluid:
\begin{itemize}[noitemsep]
	\item \texttt{updateDensity(self)}
	\item \texttt{updateMomentum(self)}
	\item \texttt{updateSpeed(self)}
	\item \texttt{streamFluid(self)}
	\item \texttt{bounceBackFluid(self)}
	\item \texttt{collideFluid(self)}
	\item \texttt{imposeVelocityBoundaryCondition(self)}
	\item \texttt{imposePressureBoundaryCondition(self)}
	\item \texttt{stepSimulation(self)}
\end{itemize}

\chapter{Optimization algorithm}

This section is still under active development and is quite incomplete both theoretically and computationally due to the time limit that has been unexpectedly imposed. Somehow, the god of AI has our back. So, enjoy the chapter!

\section{Overview}

Originally, we planned to use the stochastic gradient descent algorithm straight on the possible air conditioner position. However, it falls short because most of the best points are all scattered throughout the rooms. The time until equilibrium of most half-way point of a straight wall is mostly similar, and is the optimal position. Meaning that there are many hills and valleys in our loss function; thus, gradient descent is basically out of the window.

It was depressing. It took us four weeks before realizing that both the original gradient descent and its best cousin, \texttt{Adam}, literally doesn't work because the room's geometry. When there's a long straight wall, the function is smooth, and the local minima usually sits at the center of the straight wall. At the end of these walls are usually sharp corners, and these corners creates a spike in the function, so gradient descent wouldn't work. We hypothesized that a better algorithm would probably be a detection of all straight walls, find the midpoint of each one, then compare. However, we're already reaching the time limit of AI Builders. So that is out of the window as well. % This part must be rewritten

Recognizing that the visual geometry of the room is significant to optimizing the air conditioner placement, we settled for a convolutional neural network: a machine-learning method which can easily recognize the shape of a room. The overview of the plan is as follows:
\begin{enumerate}[noitemsep]
    \item Generate a data set that captures
          \begin{enumerate}[noitemsep]
              \item \textbf{Data}---The shape of the room in as \texttt{.png} files
              \item \textbf{Labels}---The best air conditioner position as a tuple $(Y, X)$
          \end{enumerate}
    \item Import the data into a \texttt{tensorflow} compatible format, then do train-test split
    \item Design the convolutional neural networks layer
    \item Train the model using the square of the mean squared error loss function
\end{enumerate}

\paragraph{Generation of best position} The best position is determined via brute-forcing through every possible positions that has been indexed by the \texttt{indexPosibleACPos} function and measuring the fastest time until equilibrium. The point that makes the room reaches an equilibrium the fastest is said to be \enquote{the best position.} However, this method has its own pitfalls. In that, the air conditioner is supposed to distribute the air evenly, so a more proper indexing must be
\begin{quote}
    \emph{The best air conditioner position that makes the difference of velocity of air along the boundaries the least.}
\end{quote}
However, the algorithm to determine this is too complex to be done in the time limit imposed. It will be changed to a correct one after the \texttt{AI Builders X ESCK} is done.

\paragraph{Generation of best air conditioner position} The \emph{best air conditioner placement in a room} is generated by brute-forcing through all the possible positions of air conditioner that's indexed by the function \texttt{indexPossibleACPos}.

\paragraph{Reasons for the mean-squared loss function} Since we want to predict the optimal position of the air conditioner in a room, we want the AI to minimize the distance between the prediction and the actual position. This feature is captured by the Pythagorean distance:
\begin{equation}
    d = \sqrt{(y - y')^2 + (x - x')^2}
\end{equation}
where $(y, x)$ is the actual value and $(y', x')$ is the predicted value. If we take the Pythagorean distance and square it, then throw in many data points and calculate the error between them, what we get is exactly the mean squared error.
\begin{equation}
    \bar{d^2} = \frac{1}{n}\sum_{i = 1}^{n}\ab((y_i - y'_i)^2 + (x_i - x_i')^2),
\end{equation}
where $n$ is the amount of data points that's used to validate the AI. Since the system that we are dealing with is continuous, and there are no classifications involved, the loss function directly reflects the accuracy of the model, and can be used as a metric for the model as well.

\section{Data generation algorithm}

The room generator works on a set of predetermined coordinates determined based on room size. A random amount of obstacles (Obstacles here are defined as areas that are inaccessible, whether that be furniture, columns, or wall curvature) are then placed randomly onto these positions with position harboring more than one obstacle. All the possible positions are stored in the \texttt{WallBoundary} class, and is initialized as follows:
\begin{minted}{python}
self.possiblePositions = [
    (int(self.yResolution / 3), self.xResolution - 1),
    (int(2 * self.yResolution / 3), self.xResolution - 1),
    (0, int(self.xResolution / 3)),
    (0, int(2 * self.xResolution / 3)),
    (int(self.yResolution / 3), 0),
    (int(2 * self.yResolution / 3), 0),
    (self.yResolution - 1, int(self.xResolution / 3)),
    (self.yResolution - 1, int(2 * self.xResolution / 3)),
    (0, self.xResolution - 1),
    (0, 0),
    (self.yResolution - 1, 0),
    (self.yResolution - 1, self.xResolution - 1),
]
\end{minted}

There are two types of obstacles: rectangular ones and cylindrical ones. The type of obstacle placed is determined by a random float generator, picking circles and rectangles on a ratio of 4:6 respectively.

Then, we define a function \texttt{generateRoom} in the class \texttt{WallBoundary} as follows:
\begin{minted}{python}
def generateRoom(self):
    for i in random.sample(range(12), k=random.randint(1, 12)):
        wallPos = self.possiblePositions[i]
        if random.random() < 0.4:
            if random.random() < 0.1:
                maxSize = int(min(self.yResolution, self.xResolution) * 0.7)
                minSize = int(min(self.yResolution, self.xResolution) * 0.5)
            else:
                maxSize = int(min(self.yResolution, self.xResolution) * 0.3)
                minSize = int(min(self.yResolution, self.xResolution) * 0.2)
            sizeR = random.randint(minSize, maxSize)
            self.cylindricalWall(wallPos, sizeR)
        else:
            if random.random() < 0.2:
                maxSize = int(min(self.yResolution, self.xResolution) * 0.6)
                minSize = int(min(self.yResolution, self.xResolution) * 0.3)
            else:
                maxSize = int(min(self.yResolution, self.xResolution) * 0.4)
                minSize = int(min(self.yResolution, self.xResolution) * 0.2)
            
            sizeX = random.randint(minSize, maxSize)
            sizeY = random.randint(minSize, maxSize)
            rectPos = (
                wallPos[0] - ((sizeY * WallBoundary.directions[i][0])/2),
                wallPos[1] - ((sizeX * WallBoundary.directions[i][1])/2),
            )
            endPos = (
                wallPos[0] + ((sizeY * WallBoundary.directions[i][0])/2),
                wallPos[1] + ((sizeX * WallBoundary.directions[i][1])/2),
            )
            self.filledStraightRectangularWall(rectPos, endPos)
\end{minted}

The cylindrical wall is added to the room contours using a function that takes in the center-point, and the radius. The size of the circle is picked in two steps. First, we see if we'll get a big circle or a small circle, at a chance of $40\%$ and $60\%$ respectively. This is due to the fact that, during testing, it was found that too many large circles often created small 1-pixel channels with little to no airflow. Increasing the size range of all circles didn't work, since those channels would still appear, and we would also have to increase the size of the rectangular walls to prevent creating pockets with no air-flow at all. Thus, it was decided that if a circle was going to be big, it was going to be very big, but also very rare, making it take up as much space as possible to prevent space between it and other obstacles.

The rectangular ones operate on the same principles as the cylindrical ones, only with probabilities tweaked for the average size to be smaller, as larger and longer rectangles protruding from the walls would further increase the chance of air pockets. The generator function for this takes in the top-left, and bottom-right position of the rectangles, and thus additional math is needed to convert center, height, and length into that format.

The generators are then written as a dataframe as follows:
\begin{minted}{python}
 new = pd.DataFrame(
    {
        "BestX" : [],
        "BestY" : [],
        "BestTime" : [],
        "WorstTime" : [],
        "ImageLink": [],
        "SizeX": [],
        "SizeY": [],
        "NumberOfCuts": [],
        "TypesOfCuts": [],
        "CutPositionsX": [],
        "CutPositionsY": [],
        "CutSizesX":  [],
        "CutSizesY": []
    }
)
for i in range(1000000):
    yResolution = 32
    xResolution = 32
	# Room size is fixed to 32x32
    ACspeed = 1

    walls = WallBoundary(yResolution, xResolution)
    walls.borderWall()
    conditions = walls.generateRoom()
    walls.generateIndex()
    walls.generateACDirections()
    walls.indexWithoutCare()   
\end{minted}

Then, another function is used to generate the room based on the previous function, and index all the possible positions the air conditioner can go on.
\begin{minted}{python}
for index, pos in enumerate(walls.possibleACIndex):
    print(f"Simulation batch {start}/{total}")
    simTimes = []
    for directionIndex, direction in enumerate(WallBoundary.unitVect):
        #print(f"Trying direction: {direction}...")
        pathOfFlow = [i+j for i, j in zip(pos, direction)]
        if (not walls.boundary[pathOfFlow[0], pathOfFlow[1]]):
            #print("Works! Testing now...")
            velocityInlet = [VelocityBoundary(pos[0], pos[1], ACspeed, directionIndex)]
            initCondition = np.ones((yResolution, xResolution, Simulation.latticeSize)) / 9
            simulation = Simulation(
                yResolution, xResolution, initCondition, walls, velocityBoundaries=velocityInlet, pressureBoundaries=[]
            )
            simTime, stable = simulation.simulateUntilEquilibrium(
                equilibriumThreshold=2.5, limit=500
            )
            #print("Done!")
            if not stable or simTime == 500:
                #print("It exploded :() or didn't equalize")
                break
            simTimes.append(simTime)
        #else:
            #print("Doesn't work! Moving on...")
        #print("\n")
    results.append(np.average(simTimes))
    start += 1
    #print("--------------")}
\end{minted}

Then, the generator iterate through all positions and viable (facing into open air) directions. For position, its result is saved as the average result from all viable directions.
\begin{minted}{python}
    if(len(results) == 0):
        continue

    full_frame()
    plt.imshow(walls.invertedBoundary, cmap="hot", interpolation="nearest")
    imgpath = f'data/images/{count}.png'
    plt.savefig(imgpath, bbox_inches="tight", pad_inches=0)
    # Save image of room.
   
    results = np.nan_to_num(results, copy=False, nan=501)
    # In case some positions averages equate to nan, covert to 501.
    lowest = min(results)
    most = max(results)
    ansIndex = [index for index, result in enumerate(results) if result == lowest][0]
    # Get the index with the lowest result.
    ansPos = walls.possibleACIndex[ansIndex]

    conditions["BestX"] = [ansPos[1]]
    conditions["BestY"] = [ansPos[0]]
    conditions["BestTime"] = [lowest]
    conditions["WorstTime"] = [most]
    conditions["ImageLink"] = [imgpath]

    print(f"position: {ansPos} Momentum: {results[ansIndex]}")
    df_dictionary = pd.DataFrame(conditions)
    new = pd.concat([new, df_dictionary], ignore_index=True)
    count += 1
    # Save everything to data.
\end{minted}

\section{Convolutional neural network}

We used \texttt{Tensorflow}'s Convolutional Neural Network framework to train our model. The images of the room were scaled down from their original $480\times 480$ size to a more manageable $32\times 32$ size. Its grayscale values were also inverted for more efficiency. We had generated $1438$ possible room layouts and air conditioner positions, and used it to train and test our model with a train to test ratio of $7:3$. Our model has ten layers, consisting of \texttt{Conv2D}, \texttt{MaxPooling2D}, and Dense layers arranged as follows:
\begin{enumerate}
    \item \texttt{Conv2D(32, (4, 4), activation="relu", input\_shape=(32, 32, 1))}
    \item \texttt{MaxPooling2D((2, 2))}
    \item \texttt{Conv2D(64, (4, 4), activation="relu")}
    \item \texttt{MaxPooling2D((2, 2))}
    \item \texttt{Conv2D(64, (4, 4), activation="relu")}
    \item \texttt{Flatten()}
    \item \texttt{Dense(64, activation="relu")}
    \item \texttt{Dense(32, activation="relu")}
    \item \texttt{Dense(16, activation="relu")}
    \item \texttt{Dense(2, activation="relu")}
\end{enumerate}
The first five is used to extract features that are relevant to our system, and the last five is used to calculate those features and output it into numbers while keeping computational overhead low. In the end, we used the \texttt{Adam} optimizer to increase the training efficiency, and the best error (using MSE) we could achieve was $2.56$, meaning that the predicted position was on average $1.5$ pixels away from the optimal position. 

Our baseline was roughly calculated based on the assumption that since most rooms are rectangular, ours being a square of size $32\times 32$ in particular, the minimal distance from the optimal position where cooling starts to be affected should be about $1$ side or $32$ pixels, since from optimal testing it appears the optimal placement tended towards corners. This means that the highest acceptable error is around $45$, since maximum distance where cooling time would be minimally affected. Thus, comparing this to our actual error of $2.56$, our AI is a success.

\section{Discussion}

{\small \emph{Written by Puripat Thumbanthu as strongly advised by Assoc. Prof. Dr. Wiwat Ruenglertpanyakul}}

Here, I'd like to address the pitfalls of the model. By my standards, this project is more than complete. We have built a full-fledged two-dimensional fluid simulation in less than ten weeks. From knowing literally nothing about the field to being able to understand the equation that underlies it. It is a huge achievement in terms of learning. The convolutional neural networks also worked extremely well for the room shapes in two dimensions. Too well almost. However, this project has many major pitfalls that I feel too guilty to not state it before ending this report. So, I shall address the elephant in the room here.

Firstly, the fluid simulation currently applies only partially to three-dimensional systems. Most of the airflow in a room is driven by secondary and tertiary flows, which are responsible for creating vortices. While secondary flows occur in two dimensions, tertiary flows do not. These tertiary flows contribute significantly to the complexity of simulating fluids in three dimensions. Addressing this issue would require extending the Lattice-Boltzmann method to three dimensions—a feasible task, but one that would significantly delay data processing.

Secondly, the boundary conditions in the Lattice-Boltzmann method do not fully capture air conditioner dynamics. An actual air conditioner functions as both a velocity and a pressure boundary condition. The reason a room doesn't explode under the influence of the air conditioner is that walls and windows allow air to escape. Those walls and leaks aren't included in our model yet. Instead, we rely on the assumption that a density boundary condition can approximate the effects of velocity and pressure boundary conditions, which introduces some inaccuracies. These inaccuracies are also doubled down by the fact that the function that we used are the time until fluid equilibrium, not thermal differences on the wall.

Thirdly, there isn't a standard metric that can capture the exact thermal units of an air conditioner in the Lattice-Boltzmann method. From my current knowledge, the Lattice-Boltzmann method is quite a mathematical method, and there isn't a well-defined unit for measurement. The room that we used is $32\times 32$, but the size of the density boundary condition is only $1\times 1$. We don't know the exact magnitude of the fluid vector that can imitate a home air conditioner with $12,000$ BTU. If a $1\times 1$ density boundary actually represents a square meter of air conditioning unit, let's say $12,000$ BTU, that would be like trying to cool down a closed basketball field with a single air conditioner, which is absurd. Still, we don't know if this is true or not.

Nonetheless, despite the limitations of our computational model as well as certain time constraints imposed on us, I have achieved the goal I set out to achieve during my time in this camp, and still attest to the level of quality that our model has. This would not be possible if not for P. Jenta Wonglertsakul and P. Nattavee Sunitsakul, our mentor that has guided us all the way through, and our colleague, Kritpatchara Wongkwan, for providing us with this opportunity to learn much about fluid dynamics and artificial intelligence by contacting \texttt{AI Builders} team to host this \texttt{AI Builders X ESCK} event in the first place.

\printbibliography

\end{document}
